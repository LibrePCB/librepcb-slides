\section{Footprint Assignment}

\begin{frame}{\secname}
  \textbf{Problem}
  \begin{itemize}
    \item Schematics contain "exact parts" (footprint known)
  \end{itemize}

  \pause

  \textbf{Result}
  \begin{itemize}
    \item Wasting time for choosing (irrelevant) footprints in schematics
    \item Changing footprints afterwards requires to adjust schematics
  \end{itemize}
\end{frame}

\begin{frame}{\secname}
  \textbf{Solution}
  \begin{itemize}
    \item<1-> Schematics contain "components" (footprint unknown)
    \item<2-> Boards contain "devices" (assign package to component)
  \end{itemize}
  \vspace*{-\baselineskip}\leavevmode % reduce space
  \begin{center}
    \footnotesize
    \def\fixedheight{\vrule height 0mm depth 0.7mm width 0pt} % invisible vertical line
    \begin{tikzpicture}[node distance = 8mm, text width=18mm, every fit/.style={inner sep=2mm}]

      % symbol
      \begin{pgfonlayer}{foreground}
        \node (symbol) [lpbox, rectangle split, rectangle split parts=4] {
          \fixedheight\textbf{Symbol}
          \nodepart{two}   \fixedheight Pin \textbf{1}
          \nodepart{three} \fixedheight Pin \textbf{2}
          \nodepart{four}  \fixedheight Pin \textbf{3}
        };
      \end{pgfonlayer}

      % component
      \begin{pgfonlayer}{foreground}
        \node (component) [lpbox, rectangle split, rectangle split parts=4, right=of symbol] {
          \fixedheight\textbf{Component}
          \nodepart{two}   \fixedheight Signal \textbf{GND}
          \nodepart{three} \fixedheight Signal \textbf{VCC}
          \nodepart{four}  \fixedheight Signal \textbf{OUT}
        };
      \end{pgfonlayer}
      \draw[lparrow, draw=red, ultra thick] (component.two   west) -- (symbol.three east);
      \draw[lparrow, draw=red, ultra thick] (component.three west) -- (symbol.two   east);
      \draw[lparrow, draw=red, ultra thick] (component.four  west) -- (symbol.four  east);

      % device
      \begin{pgfonlayer}{foreground}
        \onslide<2->{\node (device) [lpbox, rectangle split, rectangle split parts=4, right=of component] {
          \fixedheight\textbf{Device}
          \nodepart{two}   \fixedheight \textbf{GND} \faArrowRight\ \textbf{2}
          \nodepart{three} \fixedheight \textbf{VCC} \faArrowRight\ \textbf{3}
          \nodepart{four}  \fixedheight \textbf{OUT}  \faArrowRight\ \textbf{1}
        };}
      \end{pgfonlayer}

      % package
      \begin{pgfonlayer}{foreground}
        \onslide<2->{\node (package) [lpbox, rectangle split, rectangle split parts=4, right=of device] {
          \fixedheight\textbf{Package}
          \nodepart{two}   \fixedheight Pad \textbf{1}
          \nodepart{three} \fixedheight Pad \textbf{2}
          \nodepart{four}  \fixedheight Pad \textbf{3}
        };}
      \end{pgfonlayer}
      \draw<2->[lparrow2, draw=red, ultra thick] (component.two   east) -- (package.three west);
      \draw<2->[lparrow2, draw=red, ultra thick] (component.three east) -- (package.four  west);
      \draw<2->[lparrow2, draw=red, ultra thick] (component.four  east) -- (package.two   west);

      % schematic
      \node (txtschematic) [text centered, above=1.5cm of $(symbol)!0.5!(component)$] {\textbf{Schematic}};
      \begin{pgfonlayer}{background}
        \node[lpbox, fill=yellow!20, text=black, fit=(symbol)(component)(txtschematic)] (schematic) {};
      \end{pgfonlayer}

      % board
      \onslide<2->{\node (txtboard) [text centered, above=1.5cm of $(device)!0.5!(package)$] {\textbf{Board}};}
      \begin{pgfonlayer}{background}
        \onslide<2->{\node[lpbox, fill=yellow!20, text=black, fit=(device)(package)(txtboard)] (board) {};}
      \end{pgfonlayer}
    \end{tikzpicture}
  \end{center}

  \vspace*{-.5\baselineskip}\leavevmode % reduce space
  \onslide<3->{\faArrowRight\ Schematics can be drawn without worrying about footprints}\\
  \onslide<4->{\faArrowRight\ Footprints can easily be changed in boards}\\
  \onslide<5->{\faArrowRight\ Pinout (pin-to-pad map) stored in library}
\end{frame}
