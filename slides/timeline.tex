\section{Timeline}

\begin{frame}{\secname}

  \bigskip
  \footnotesize
  \begin{chronology}[startyear=2022,stopyear=2024]
    \chronoevent[datesstyle=\it,conversionmonth=false,datesseparation=.,
      colorbox=white,markdepth=20pt]{4/2/2024}{FOSDEM'24}
    \chronoevent[datesstyle=\it,conversionmonth=false,datesseparation=.,
      colorbox=white,markdepth=20pt]{24/9/2023}{v1.0}
    \chronoevent[datesstyle=\it,conversionmonth=false,datesseparation=.,
      colorbox=white,markdepth=60pt]{21/8/2023}{v1.0-rc1}
    \chronoevent[datesstyle=\it,conversionmonth=false,datesseparation=.,
      colorbox=white,markdepth=20pt]{13/4/2023}{NGI0 Grant}
    \chronoevent[datesstyle=\it,conversionmonth=false,datesseparation=.,
      colorbox=white,markdepth=20pt]{3/10/2022}{v0.1.7}
    \chronoevent[datesstyle=\it,conversionmonth=false,datesseparation=.,
      colorbox=white,markdepth=60pt]{8/9/2022}{PCBWay Partnership}
    \chronoevent[datesstyle=\it,conversionmonth=false,datesseparation=.,
      colorbox=white,markdepth=20pt]{5/2/2022}{FOSDEM'22}
  \end{chronology}

\end{frame}

\note{
  Two years ago I had the last project update talk at FOSDEM 20. Since then,
  we released three new versions of LibrePCB, which add many new features and
  improvements.\\

  Last year we also switched from the qmake build system to CMake and refactored
  the software architecture to keep the project maintainable and future-proof.\\

  I will not mention every improvement here, for that you could read the full
  changelog on our website.\\

  Instead, I picked just a few of the improvements which I will show you here.
}
