\section{Project Status}

\begin{frame}{\secname}
  \bigskip
  \begin{table}
    \large\bf
    \begin{tabular}{r m{0.5cm} l}
      Library Management & \Smiley[1.8][green] & \\
      Library Editor & \Smiley[1.8][green] & \\
      Schematic Editor & \Smiley[1.8][green] & (except buses \& hierarchical sheets) \\
      Board Editor & \Neutrey[1.8][yellow] & (functional, but rather basic editing tools) \\
      Data Import & \Neutrey[1.8][yellow] & \\
      Data Export & \Smiley[1.8][green] &  \\
      Available Libraries & \Neutrey[1.8][yellow] &  \\
    \end{tabular}
  \end{table}
\end{frame}

\note{
  Now, what's the overall state of this project?\\

  Generally LibrePCB is fully functional and can be used productively for
  projects which are not too complex. Not too complex because hierarchical
  schematics and buses are not supported yet. Also the trace routing
  or the board editor in general is still rather rudimentary, so from time
  to time it might be a bit inefficient.\\

  Also the part libraries are not very comprehensive yet, but of course you can
  create any missing part easily by yourself.
}
